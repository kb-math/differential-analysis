\documentclass[twoside, a4paper, 10pt]{amsart}
\title[ ]{Taylor's theorem in normed vector spaces}
\usepackage{amsaddr}
\author{kb-math}
\usepackage{amsfonts}
\usepackage{amsthm}
\usepackage{verbatim}
\usepackage{amsmath, amssymb}
\usepackage{tikz}
\usetikzlibrary{matrix, arrows}
\usepackage{listings}
\usepackage{color}
\usepackage{listings}
\usepackage[all]{xy}
\usepackage[pdftex,colorlinks,linkcolor=blue,citecolor=blue]{hyperref}
\usepackage{graphicx}
\usepackage{float}
\usepackage[margin=3cm]{geometry}
\usepackage{bigints}
\usepackage{dsfont}
\setlength{\textwidth}{6.5in}
\setlength{\oddsidemargin}{0in}
\setlength{\evensidemargin}{0in}
\setlength{\parindent}{0pt}
\setlength{\parskip}{1ex plus 0.5ex minus 0.2ex}
\linespread{1.3}
\begin{document}
\maketitle
\raggedbottom


%% Mathcal large
\newcommand{\cA}{\mathcal{A}}
\newcommand{\cB}{\mathcal{B}}
\newcommand{\cC}{\mathcal{C}}
\newcommand{\cD}{\mathcal{D}}
\newcommand{\cE}{\mathcal{E}}
\newcommand{\cF}{\mathcal{F}}
\newcommand{\cG}{\mathcal{G}}
\newcommand{\cH}{\mathcal{H}}
\newcommand{\cI}{\mathcal{I}}
\newcommand{\cJ}{\mathcal{J}}
\newcommand{\cK}{\mathcal{K}}
\newcommand{\cL}{\mathcal{L}}
\newcommand{\cM}{\mathcal{M}}
\newcommand{\cN}{\mathcal{N}}
\newcommand{\cO}{\mathcal{O}}
\newcommand{\cP}{\mathcal{P}}
\newcommand{\cQ}{\mathcal{Q}}
\newcommand{\cR}{\mathcal{R}}
\newcommand{\cS}{\mathcal{S}}
\newcommand{\cT}{\mathcal{T}}
\newcommand{\cU}{\mathcal{U}}
\newcommand{\cV}{\mathcal{V}}
\newcommand{\cW}{\mathcal{W}}
\newcommand{\cX}{\mathcal{X}}
\newcommand{\cY}{\mathcal{Y}}
\newcommand{\cZ}{\mathcal{Z}}
%% Mathbb large
\newcommand{\bA}{\mathbb{A}}
\newcommand{\bB}{\mathbb{B}}
\newcommand{\bC}{\mathbb{C}}
\newcommand{\bD}{\mathbb{D}}
\newcommand{\bE}{\mathbb{E}}
\newcommand{\bF}{\mathbb{F}}
\newcommand{\bG}{\mathbb{G}}
\newcommand{\bH}{\mathbb{H}}
\newcommand{\bI}{\mathbb{I}}
\newcommand{\bJ}{\mathbb{J}}
\newcommand{\bK}{\mathbb{K}}
\newcommand{\bL}{\mathbb{L}}
\newcommand{\bM}{\mathbb{M}}
\newcommand{\bN}{\mathbb{N}}
\newcommand{\bO}{\mathbb{O}}
\newcommand{\bP}{\mathbb{P}}
\newcommand{\bQ}{\mathbb{Q}}
\newcommand{\bR}{\mathbb{R}}
\newcommand{\bS}{\mathbb{S}}
\newcommand{\bT}{\mathbb{T}}
\newcommand{\bU}{\mathbb{U}}
\newcommand{\bV}{\mathbb{V}}
\newcommand{\bW}{\mathbb{W}}
\newcommand{\bX}{\mathbb{X}}
\newcommand{\bY}{\mathbb{Y}}
\newcommand{\bZ}{\mathbb{Z}}


\newcounter{dummy} \numberwithin{dummy}{section}

\theoremstyle{definition}
\newtheorem{mydef}[dummy]{Definition}
\newtheorem{prop}[dummy]{Proposition}
\newtheorem{corol}[dummy]{Corollary}
\newtheorem{thm}[dummy]{Theorem}
\newtheorem{lemma}[dummy]{Lemma}
\newtheorem{eg}[dummy]{Example}
\newtheorem{notation}[dummy]{Notation}
\newtheorem{remark}[dummy]{Remark}
\newtheorem{claim}[dummy]{Claim}
\newtheorem{Exercise}[dummy]{Exercise}
\newtheorem{question}[dummy]{Question}

\section{Basic setup}

Let $U,V$ be normed vector spaces and suppose that $X \subset U$ is open. Let $\mathcal{L}(U,V)$ denote the normed vector space of bounded linear maps.

\begin{mydef} 

We say that $f: X \to V$ is differentiably at $x_0 \in X$ if there exists a bounded linear map $D: U \to V$ and a continuous $\epsilon: [0,\infty) \to [0, \infty)$ with $\epsilon(0) = 0$ such that $$\| f (x_0+u) - f(x_0) - Du \| \leq  \epsilon( \|u \|) \|u \|.$$

\end{mydef}

\begin{lemma} Such a $D$ must satisfy $$ Du = \lim_{t \to 0} \frac{f(x_0+tu) - f(x_0)}{t \|u \|} $$ for each fixed non-zero $u \in U$ and is hence unique. \end{lemma}

\begin{proof} For each fixed $u \in U \setminus \{0 \} $ and sufficiently small $t$ we have that $$\| f(x_0+tu) - f(x_0) - D(tu) \| = \epsilon(t \| u \|) \|t u \|. $$ Now divide both sides by $|t|$ and use the linearity of $D$ to get the estimate $$ \| \frac{f(x_0+tu) - f(x_0)}{t } - Du \| \leq \epsilon(t \|u \|) \| u \|.$$  The result now follows by letting $t \to \infty$. \end{proof}

Given the uniqueness, we can now define $D = Df(x_0) \in \mathcal{L}(U,V)$ to be the derivative of $f$ at $x_0$. 

\begin{lemma} If $f:X \to V$ is differentiable at $x_0$ then it is continuous at $x_0$. \end{lemma}

\begin{proof} By triangle inequality we have that $$\| f(x_0 + u) - f(x_0) \| \leq  \| f(x_0 + u ) - f(x_0) - Df(x_0) u \| + \|Df(x_0)u \|$$ but both terms on the right hand side converge to zero as $u \to 0$ by the differentiability of $f$ and the boundedness of $Df(x_0)$, respectively. \end{proof}

\section{Higher order derivatives}

How do we define higher order derivatives in this setting? If $f:X \to V$ is differentiable on $X$ then $Df: X \to \mathcal{L}(U,V)$. Now $\mathcal{L}(U,V)$ is itself a normed vector space (equipped with operator norm) hence the derivative of $Df$ (assuming it exists) is a map $$D^2 f: X \to \mathcal{L}(U, \mathcal{L}(U,V)).$$ This means that given $u_1, u_2 \in U$ we have that $$D(Df)(x_0) \in \mathcal{L}(U,\mathcal{L}(U,V))$$ and hence $$(D(Df)(x_0) )u_1 \in \mathcal{L}(U,V)$$ and hence $$((D(Df)(x_0))u_1)u_2 \in V.$$ We abbreviate this element in $V$ by $$(D^2f)  (x_0) (u_1, u_2)$$ and hence we get a \textbf{multilinear map} $$(D^2 f) (x_0):U \times U \to V$$ called the \textit{second derivative} of $f$.

\begin{lemma}\label{lemma: second derivative is iterated derivative} If $f: \mathbb{R}^n \to V$ is twice differentiable then $$ D^2 f (x_0) (e_i, e_j) = \partial_i \partial_j f(x_0)$$ where $\partial_i$ is partial differentiation with respect to $i$-th co-ordinate. \end{lemma}

This can be generalized to higher order derivatives, hence $D^n f(x_0) : U \times \cdots \times U \to V$ may be realised as a multilinear map, if it exists, and it can be evaluated through iterated directional derivatives. To make this precise it is worth formalizing some notions regarding the continuity of multilinear maps.

\begin{mydef} Given normed vector spaces $U_1, \ldots, U_n$ and $V$, let $\mathcal{T}(U_1, \ldots, U_n \to V)$ denote the vector space of \textit{bounded} multilinear maps $T: U_1 \times \cdots \times U_n \to V$, where bounded means that $$\| T \| := \sup \{ T(u_1, \ldots, u_m) ~|~ u_i \in U_i \text{ with } \|u_i \| = 1 \} < \infty.$$ Moreover, $\| \cdot \|$ is a norm on the space of bounded multilinear maps.  \end{mydef}

\begin{prop} There is a canonical isometric isomorphism $$ \mathcal{L}(U_1, \mathcal{T}(U_2, \ldots, U_n \to V)) \cong \mathcal{T}(U_1, \ldots U_n \to V)$$ given by $$\phi \mapsto ((u_1, \ldots, u_n) \mapsto \phi(u_1)(u_2, \ldots, u_n)).$$  \end{prop}

In other words, we could have defined $\| T \|$ as the metric induced by the canonical isomorphism $$ \mathcal{T}(U_1, \ldots, U_n \to V) \cong  \mathcal{L}(U_1, \ldots \mathcal{L}(U_{n-1}, \mathcal{L}(U_n,V)) \ldots) $$

\begin{prop} A multilinear map $T: U_1 \times \cdots \times U_n \to V$ is bounded if and only if it is continuous with respect to the product topology. \end{prop}

Hence the nth derivative $D^nf : X \to \mathcal{T}(U, \ldots, U \to V)$ may be recursively defined as $D^1f = Df$ (which makes sense since $\mathcal{T}(U \to V) = \mathcal{L}(U, V)$) and $$D^{(n+1)}f (x_0)(u_1, \ldots, u_{n+1}) = [(D(D^nf))(x_0)u_1] (u_2, \ldots, u_{n+1})$$ if it exists (and if, of course, $D^nf$ exists on $X$). 

\section{Basic properties}

\begin{lemma}(Chain Rule) Suppose that $U,V,W$ are normed vector spaces with $X \subset U$ and $Y \subset V$ open sets. Suppose $f:X \to Y$ is differentiable at $x_0 \in X$ and $g:Y \to W$ is differentiable at $y_0 = f(x_0) \in Y$. Then $g \circ f:X \to W$ is differentiable at $x_0$ and $$D(g \circ f)(x_0) = D(g)(y_0) \circ D(f)(x_0).$$ \end{lemma}

Given normed vector spaces $U_1, U_2$ we will always equip the direct sum with the max norm $\|(u_1, u_2)\| = \max \{\|u_1\|, \|u_2 \| \}$, which induces the product topology.

\begin{lemma}(Product Rule for multilinear maps) Suppose that $U_1, \ldots, U_n$ are normed vector spaces and $T: U_1 \times \cdots \times U_n \to V$ is a bounded multilinear map. Then $T$ is differentiable at each $\vec{x} = (x_1, \ldots, x_n) \in U_1 \times \cdots \times U_n $ with $$D(T)(\vec{x})(u_1, \ldots, u_n) = T(u, x_2, \ldots, x_n) + T(x_1, u, x_3, \ldots, x_n) + \cdots T(x_1, \ldots x_{n-1}, u). $$\end{lemma}

\begin{lemma}(General product rule)  Suppose that $U_1, \ldots, U_n$ are normed vector spaces and $T: U_1 \times \cdots \times U_n \to V$ is a bounded multilinear map and suppose that $f_i:X \to U_i$ are differentiable at $x_0 \in X$. Then the map $$ H(x) = T(f_1(x), \ldots f_n(x)) $$ is differentiable at $x_0$ and has derivative $$DH(x_0)u = T((Df_1)(x_0)u, f_2(x_0), \ldots, f_n(x_0)) + \cdots + T(f_1(x_0), f_2(x_0), \ldots, Df_n(x_0)u)$$\end{lemma}

\begin{proof} The map $F(x)=(f_1(x), \ldots, f_n(x))$ has derivative $$DF(x_0)u =(Df_1(x_0)u, \ldots, Df_n(x_0)u)$$ and $H = T \circ F$ hence the derivative is 

\begin{align*} D(H)(x_0)u &= D(T)(F(x_0))(DF)(x_0)u \\ &= DT(F(x_0)) (Df_1(x_0)u, \ldots, Df_n(x_0)u)  \end{align*}

and the result now follows from the product rule for multilinear maps.

 \end{proof}

This example will be important for our proof of Taylor's theorem and can be used to give a higher order product rule.
\begin{eg} \label{product rule for compositions} Consider the composition map $$ \mathcal{L}(U,V) \times \mathcal{L}(W, U)  \to \mathcal{L}(W,U)$$ given by $(L_1, L_2) \mapsto L_1 \circ L_2$. Then it is a bounded bilinear map. Now suppose that we have maps $g: X \to \mathcal{L}(W,U)$ and $h: X \to \mathcal{L}(U,V)$ differentiable at $x_0 \in X$. Then the map $M(x) = h(x) \circ g(x)$ is differentiable at $x_0$ with derivative given by $$ DM(x_0)u = D(h(x_0))u \circ g(x_0) + h(x_0) \circ Dg(x_0)u $$ \end{eg}

\begin{lemma}(Mean value inequality) Suppose $f:X \to V$ is differentiable on each point of the segment $x_0 + tu$ for $t \in [0,1]$ and that $t \mapsto Df(x_0 + tu)$ is continuous on $[0,1]$ . Then $$\| f(x_0 + u) - f(x_0) \| \leq \sup_{t \in [0,1]} \| Df(x_0 + tu) u \|.$$ \end{lemma}

\begin{proof} We first reduce to the case $V = \mathbb{R}$ as follows. By the Hahn-Banach theorem we have a linear $\phi:V \to R$ such that $\phi (f(x_0 + u) - f(x_0)) = \| f(x_0 + u) - f(x_0) \|$ and $|\phi \| = 1$. So if the theorem is true for $g = \phi \circ f: X \to V$ we must have $$\|f(x_0+ u) -f(x_0)\| = \|g(x_0 + u) - g(x_0) \| \leq \sup_{t \in [0,1]} \| Dg(x_0 + tu) u \|.$$ But by the chain rule and the fact that $D\phi(x) = \phi$ for all linear $\phi$, we have that $Dg = \phi \circ Df$. Hence we have that $$\|Dg(x_0 +tu)u \| = \| \phi Df(x_0+tu)u \| \leq Df(x_0+tu)u.$$ 

We now turn to show that the theorem holds when $V = \mathbb{R}$. Let $h(t) = f(x_0+tu)$. Note that the classical derivative $h'(t) = Dh(t) (1)$ is continuous, hence we may apply the fundamental theorem of calculus to get $$\| f(x_0 + u) - f(x_0) \| = |h(1) - h(0)| \leq |\int_{0}^1 h'(t) dt| \leq \sup_{t \in [0,1]}|h'(t)|.$$ But from the chain rule we get
\begin{align*} h'(t_0) &= Dh(t_0)(1) \\ &= D(f(x_0 + ut_0)) D(t \mapsto (x_0 + tu))(t_0)(1) \\ &= D(f(x_0+t_0u)) (t \mapsto tu)(1) \\&=D(f(x_0+t_0u)) u  \end{align*} \end{proof}

\section{The commutativity of higher order derivatives}

\begin{thm} Suppose that $f:U \to V$ is twice differentiable at $x_0$, then $D^2 f(x_0)$ is a symmetric bilinear form, that is $$D^2f(x_0)(u_1,u_2) = D^2f(x_0)(u_2,u_1) \quad \text{for all } u_1,u_2 \in U.$$  \end{thm}

\begin{proof} It is sufficient to show this for all small enough $u_1, u_2 \in U$. Let $g(x) = f(x + u_1) - f(x)$. If $L:U \to V$ is the bounded linear map $L = D^2f(x_0)(u_1)$ then by the mean value inequality applied to $g - L$ we have that \begin{align}\label{mean value second derivative} \| g(x_0 + u_2) - g(x_0) - L(u_2) \| \leq \sup_{t \in [0,1]} \| Dg(x_0 + tu_2)u_2 - L(u_2) \| \end{align} for all small enough $u_1, u_2 \in U$. 

By the definition of differentiability we have that $$Df(x_0 + u) = Df(x_0) + D^2f(x_0)(u) + R(u) $$ where $\| R(u) \| \leq \epsilon(u)\|u \|$ where $ \lim_{u \to 0} \epsilon(u) = 0$. We use this inequality to estimate the upper bound in (\ref{mean value second derivative}) as follows

\begin{align*} Dg(x_0 + tu_2)u_2 &= Df(x_0 + tu_2 + u_1)u_2 - Df(x_0 +tu_2)u_2 \\
 &= D^2f(x_0) (tu_2 + u_1)u_2 - D^2f(x_0)(tu_2)u_2 + R(tu_2 + u_1)u_2 - R(tu_2)u_2 \\ &= D^2f(x_0)(u_1)(u_2) + R(tu_2 + u_1)u_2 - R(tu_2)u_2  \end{align*} Hence (\ref{mean value second derivative}) gives the estimate \begin{align*} \| g(x_0 + u_2) - g(x_0) - D^2f(x_0)(u_1, u_2) \| & \leq \sup_{t \in [0,1]} \|  R(tu_2 + u_1)u_2 - R(tu_2)u_2 \| \\ & \leq \epsilon_0(\|u_1\| + \|u_2\|) (\|u_1 \| +  \|u_2 \|)^2  \end{align*} for some $\lim_{u \to 0} \epsilon_0(\|u \|) = 0$. In particular, by replacing $u_i$ with $s u_i$ for $0< s < 1$ and dividing by $s^2$ we get the limit \begin{align*} D^2f(x_0)(u_1)(u_2) &= \lim_{s \to 0} \frac{g(x_0 + su_2) - g(x_0)}{s^2} \\ &= \lim_{s \to 0} \frac{f(x_0 + su_1 + su_2) - f(x_0 + su_2) - f(x_0 + su_1) + f(x_0)}{s^2} \end{align*} but this term inside the limit is symmetric in $u_1$ and $u_2$. \end{proof}

We can apply this inductively to get (notice that we don't require $D^nf(x)$ to be defined or continuous on any open set): 
\begin{thm} Suppose that $f:U \to V$ is such that $D^nf(x_0)$ exists at $x_0 \in U$, then $D^nf(x_0)$ is a symmetric multilinear form. \end{thm} 

\section{Taylor's Theorem}

If $h: \mathbb{R} \to \mathbb{R}$ is $N$ times continuously differentiable at $x_0 \in \mathbb{R}$ then the classical Taylor expansion is $$h(x_0 + u) = h(x_0) + h'(x_0)u + \frac{1}{2!}h^{(2)}(x_0)u^2 + \cdots + \frac{1}{N!}h^{(N)}(x_0)u^N + R(u)$$ with error term $$R(u) = \frac{1}{(N+1)!} \int_{0}^u h^{(N+1)}(x_0 + t) t^n dt.$$ 

We now ask how can this be formulated for $f:U \to \mathbb{R}$ where $U$ is an arbitrary vector space? If we define $u^n$ to be the $n$-tuple $(u, \ldots, u)$ then the expression $Df^n(x_0) u^n$ makes sense, as $Df^n(x_0)$ is a multilinear map. Hence a strategy of deriving a Taylor theorem involves first parametrizing the line segment from $x_0$ to $x_0 + u$ by defining $L(t) = tu$ and applying the classical Taylor theorem to $h = f \circ L$. To do this we need to compute $D^n(f \circ L) (t_0)$. 

\begin{lemma}\label{lemma: iterated linear chain rule}  Let $L: W \to U$ be a linear map and let $f:U \to V$ be $n$ times differentiable at $L(a)$, where $a \in W$. Then $$D^n(f \circ L)(a) = D^nf (L(a)) \circ  L^n$$ where $L^n(w_1, \ldots, w_n) := (Lw_1, \ldots, Lw_n)$. \end{lemma}

\begin{remark} The is a generelization of the standard identity $h^{(n)}(at) = a^n h^{(n)}(at)$ from classical differential calculus. \end{remark}

\begin{proof} For $n=1$ this is the chain rule and the fact that $D(L)(a)=L$ for linear maps $L$. Now we proceed by induction. Assuming the result is true for $n$, we now suppose that $f$ is $n+1$ differentiable at $L(a)$. Then we get

\begin{align*} D^{n+1}(f \circ L)(a)  &= D(D^n(f \circ L))(a)) \\ &= D( w \mapsto D^n f(Lw)  \circ L^n)(a). \end{align*} 

To compute this term we use the product rule, specifically Example \ref{product rule for compositions} to the map $w \mapsto D^nf(Lw)$ and the constant map $w \mapsto L^n$. The latter has zero derivative so we only get the first term from Example ((\ref{product rule for compositions})). This means that \begin{align*} D^{n+1}(f \circ L)(a) (w_0, w_1, \ldots, w_n) &= D^{n+1}(f \circ L)(a)(w_0) (w_1, \ldots, w_n) \\ &=( D( w \mapsto D^nf(Lw)) (a) (w_0) \circ L^n ) (w_1, \ldots, w_n) \end{align*}

where we firstly used our canonical isomorphism  $$\mathcal{L}(W_0, \mathcal{T}(W_1, \ldots, W_n \to V)) \cong \mathcal{T}(W_0, \ldots, W_n \to V)$$ with $W_i = W$ followed by Example \ref{product rule for compositions} as described.

 Now to compute this final derivative we use the chain rule as follows. Let $F(u) = D^nf(u)$. Then this final derivative is $D(F \circ L) (a)$ hence equal to $D(F)(La) \circ L= D^{n+1}f(La) \circ L$. So we have shown that \begin{align*} D^{n+1}(f \circ L)(a)(w_0, \ldots, w_n) &= \left((D^{n+1}f(La) \circ L)w_0 \circ L^n) \right) (w_1, \ldots, w_n) \\ &= D^{n+1}f(La) (Lw_0) (Lw_1, \ldots, Lw_n) \end{align*} and the proof is complete once one again applies the canonical isomorphism.
\end{proof}

\begin{thm}(Co-ordinate free general Taylor theorem) Suppose $f:X \to \mathbb{R}$ is $(n+1)$ times continuously differentiable and let $x_0 \in X$ and $u \in U$ be such that the line segment $x_0 + tu$ is in $X$. Then we have that $$f(x_0 + u) = f(x_0) + \sum_{n=1}^N D^n f(x_0) u^n + R(u)$$ where $u^n = (u, \ldots, u) \in U^n$ and $$R(u) =\frac{1}{(N+1)!} \int_{0}^1 f^{(N+1)}(x_0 + tu)u^{n+1} t^n dt  .$$  \end{thm}

\begin{proof} By translation invariance of derivatives, it is enough to prove this for $x_0 = 0$. Now write $f(x_0 + tu) - f(x_0) = h(1) - h(0)$ and use Taylor's theorem, where $h = f \circ L$ where $L:\mathbb{R} \to U$ is given by $L(t) = tu$. The result now follows from the Lemma~\ref{lemma: iterated linear chain rule} 
\end{proof}

\section{Recovering the co-ordinate based Taylor theorem}

To recover the Taylor theorem most often presented in undergraduate vector calculus, one needs to expand $D^nf(x_0) u^n$ as follows. Writing $$u = \sum_i u_i e_i$$ where $e_1, \ldots, e_n$ is a basis for $U$, we have that $$D^nf(x_0) (e_{i_1}, \ldots, e_{i_n}) = \partial_{i_1} \cdots \partial{i_n}f(x_0)$$ by the discussion surrounding Lemma~\ref{lemma: second derivative is iterated derivative} on iterated derivatives, where $\partial_i$ is partial differentiation with respect to the $e_i$ direction. Hence by multilinearity we get that $$D^nf(x_0)u = \sum_{i_1, \ldots, i_n} \partial_{i_1} \cdots \partial_{i_n}f(x_0) u_{i_1} \ldots u_{i_n}.$$ Now by commutativity of derivatives the order of $i_1, \ldots, i_n$ does not matter, hence it pays to express this in terms of partitions as follows. Given $\alpha = (a_1, \ldots, a_n) \in \mathbb{Z}_{\geq 0}^n$ such that $\sum_i a_i = n$ we define $$D^{\alpha}f(x_0) = \partial_1^{a_1} \cdots \partial_n^{a_n} f(x_0)$$ where $\partial_i^{k}$ means $\partial_i$ applied $k$ times. Define also $$u^{\alpha} = \prod_{i} u_i^{a_i}.$$ Then by grouping the terms together we get the expression $$D^n f(x_0) u = \sum_{|\alpha| = n} (\alpha!) D^{\alpha}f(x_0) u ^{\alpha}$$ where $|\alpha| = a_1 + \cdots + a_n$ and $\alpha! = a_1! \cdots a_n!$. Finally, this means that the Taylor theorem may be written as $$ f(x_0 + u ) = f(x_0) + \sum_{n=1}^N \sum_{|\alpha| = n} \binom{n}{\alpha}^{-1} D^{\alpha}f(x_0) u^{\alpha}  + R(x) $$ where $$ \binom{n}{\alpha} = \frac{n!}{\alpha!} $$ is the multinomial coefficient.

\section{Sufficient conditions for differentiability}

It is all well and good proving abstract theorems about derivatives, but so far we don't have a mechanism for quickly telling whether a certain function is differentiable. Fortunately we have the following criteria. 

\begin{prop} Suppose that $X \subset \mathbb{R}^n$ is open and $f:X \to V$ has the property that all its partial derivatives exist and are continuous. Then $f$ is continuously differentiable on $X$.

\end{prop}

\begin{proof} The hypothesis says that each limit $$\partial_if(x) = \lim_{t \to 0} \frac{f(x+te_i)}{t}$$ exists and is continuous as a function of $x$. To show differentiability at $x_0 \in X$, we first construct the derivative $D= Df (x_0) $ and then show it satisfies the definition. Without loss of generality, we will use the $1$-norm $\| \cdot \| = \| \cdot \|_1$ for convenience. We define $D$ to be the linear map $$Du = \sum_{i=1}^n u_i \partial_i f (x_0)$$ for $u = \sum_i u_i e_i$. To show that $D$ satisfies the definition of the derivative fix $\epsilon >0$ and find a corresponding $\delta >0$ such that $\| \partial_i f(x) - \partial_if (x_0) \| < \epsilon$ for all $x$ such that $\|x - x_0\| < \delta$. Choose $u$ such that each $\|u_ie\| < \frac{1}{n} \delta$ and let $w_i = u_1 + \ldots + u_i$ and $w_0 = 0$ (notice that this ensures each $\|w_i \| < \delta$). We have 

\begin{align*} f(x_0 + u) - f(x_0) &= \sum_{i=1}^n f(w_i) - f(w_{i-1}) \\ &= \sum_{i=1}^n f(w_{i-1} + u_i e_i) - f(w_{i}) \\ &= \sum_{i=1}^n (u_i \partial_i f(w_{i-1}) + R_i) \\ &= Du + \sum_{i=1}^n u_i \left(\partial_i f(w_{i-1}) - \partial_i f(x_0) \right) + \sum_{i=1}^n R_i \end{align*}

where the remainder term $R_i$ satisfies $$\|R_i \| \leq \sup_{t \in [0,1]} \| u_i \partial_i f_i (w_{i-1} + tu_ie_i) - u_i \partial_i f_i (w_0)\| \leq 2 \epsilon \|u_i\|$$ by the mean value inequality applied to the function $u \mapsto f(w_{i-1} + ue_i)$. So altogether we have shown that if $\|u_i e_i\| < \frac{1}{n}\delta$ then we have $$\|f(x_0 + u) -f(x_0) - Du\| \leq 3\epsilon \|u\|$$ which verifies differentiability. \end{proof}

This generalizes to higher order differentiability. We will focus on finite dimensional spaces. If $U$ has basis $e_1, \ldots, e_n$ and $V$ has basis $b_1, \ldots, b_m$ then the corresponding standard basis for $\mathcal{L}(U,V)$ is given by $\delta_{i,j}$ for $1 \leq i \leq m$, $1 \leq j \leq n$ where $$\delta_{i,j}\left( \sum_{j} u_j e_j \right) = u_j b_i.$$ More generally, the space of $k$-multilinear map $T:U \times \cdots \times U \to V$ has a standard basis where the elements are indexed by $i, j_1, \ldots, j_k$ where such an element maps  $(e_{j_1}, \ldots, e_{j_n})$ to $b_i$ but all other tuples in $\{e_1, \ldots, e_n\}^k$ map to $0$.

\begin{thm}  Suppose that $X \subset U = \mathbb{R}^n$ is open and $f:X \to V = \mathbb{R}^m$ has the property that all its $n$-order partial derivatives exist and are continuous. Then $f$ is continuously $n$ times continuously differentiable on $X$. \end{thm} 

\begin{proof} We have shown the $n=1$ case. Proceeding by induction, suppose that it holds for $n$. Suppose that $f:X \to \mathbb{R}^m$ has the property that all its $n+1$-order partial derivatives exist and are continuous. The induction hypothesis says that $D^nf: X \to \mathcal{T}( U, \ldots, U \to V)$ exists and is continuous. But with respect to the standard basis for $\mathcal{T}( U, \ldots, U \to V)$ the coefficients of $D^nf(x_0)$ are $n$-order partial derivatives of $f$ at $x_0$ and their partial derivatives are the order $(n+1)$ partial derivatives, which exist and are continuous by hypothesis. Applying the already shown $n=1$ case establishes the inductive step.

\end{proof}

%\begin{thebibliography}{99}

%\end{thebibliography}




\end{document}